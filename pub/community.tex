%f=linux; latex $f.tex; bibtex $f; latex $f.tex; latex $f.tex; dvips -t letter $f; ps2pdf $f.ps; evince $f.pdf
\documentclass{sig-alternate-05-2015}
%\usepackage{verbatim}
%\usepackage{comment}
\usepackage{epstopdf}

\newtheorem{observation}{\bf Observation}
\newtheorem{question}{\bf Question}
\begin{document}
% Copyright
\setcopyright{acmcopyright}
%\setcopyright{acmlicensed}
%\setcopyright{rightsretained}
%\setcopyright{usgov}
%\setcopyright{usgovmixed}
%\setcopyright{cagov}
%\setcopyright{cagovmixed}

\title{Code contribution practice in Linux kernel:
How product structure shapes organization}

\numberofauthors{4}
\author{
\alignauthor
Ben Trovato\titlenote{Dr.~Trovato insisted his name be first.}\\
       \affaddr{Institute for Clarity in Documentation}\\
       \affaddr{1932 Wallamaloo Lane}\\
       \affaddr{Wallamaloo, New Zealand}\\
       \email{trovato@corporation.com}
% 2nd. author
\alignauthor
G.K.M. Tobin\titlenote{The secretary disavows
any knowledge of this author's actions.}\\
       \affaddr{Institute for Clarity in Documentation}\\
       \affaddr{P.O. Box 1212}\\
       \affaddr{Dublin, Ohio 43017-6221}\\
       \email{webmaster@marysville-ohio.com}

\maketitle
\begin{abstract}

\end{abstract}
\keywords{product structure, code commitment organization}

\section{Introduction}
%If I tell you to think of an open-source project, the first word that probably comes to mind is Linux.

%
Linux kernel has been a legend in the Free/Libre and Open Source Software (FLOSS) world.
Linux kernel based distributions (operating systems) established a commercial success
that many other FLOSS projects would like to pursue.
For example, they take 98.8\% positions in the top 500 fastest supercomputers
in Nov 2015\footnote{https://en.wikipedia.org/wiki/Usage\_share\_of\_operating\_systems}.
The statistics from monitoring a substantial number of web sites during the
last twelve months (until Dec 2015) show Linux kernel based
web clients have a 28.89\% share in the market.

The success of Linux kernel can not be achieved without the contributions
from participants. Starting from Linus Torvards,  volunteers
played a crucial role in the development of linux kernel.
However, the number of volunteers (unpaid developers) contributing to the
Linux kernel has been slowly declining for many years, now sitting at just
 12.4\% (it was 13.6\% in 2014, and 14.6\% in 2013).
Meanwhile, commercial participation substantially grows in recent years,
large companies like RedHat and Intel put substantial resources on the development
of Linux kernel, e.g., Intel contributed 10.5\% of changes, ReHat 8.4\%
in 2014\footnote{https://s3.amazonaws.com/storage.pardot.com/6342/120970/lf\_pub\_whowriteslinux2015.pdf}.
Now more than ever, the development of the Linux kernel is a matter for
the professionals, as unpaid volunteer contributions to the project reached their
lowest recorded levels in the latest ``Who Writes Linux''
report\footnote{http://linux.slashdot.org/story/15/02/18/1745246/torvalds-people-who-start-writing-kernel-code-get-hired-really-quickly}.

\begin{comment}
As for why Linux is now mostly developed by well-paid engineers, the
possible reasons are myriad. The most obvious and compelling reason
is that these big companies have a commercial interest in the continued
good health of Linux. 10 years ago, Linux was the plaything of hobbyists
and supercomputer makers -- today, it powers everything from smartphones
(Android) to wireless routers to set-top boxes. The continuing commercial
interest in Linux is highlighted by another statistic from The Linux
Foundation report: In mid-2011, only 191 companies were involved in the
Linux kernel; by the end of 2013, that number was up to
243\footnote{www.extremetech.com/computing/175919-who-actually-develops-linux-the-answer-might-surprise-you}.
%If the Linux kernel was to survive, it would need new programmers to fix all of
%the bugs that were added\footnote{eudyptula-challenge.org}.
\end{comment}

Apparently, Linux kernel experienced dramatic change since the very
beginning, e.g., expanding of substantial code and increasing of commercial participation.
Different modules of Linux kernel present different nature
and attract different contributors. For example, the module of drivers
accounts for the largest proportion of linecount (56.6\%) in the kernel,
probably because various of
hardware manufacturers have been trying to get their drivers into the kernel
and devoted substantial effort for their cause.
 As for the module of kernel, the very central one, attracting the most
capable and ambitious developers in the world, takes only 1.2\%.
According to Conway's law~\cite{conway}, the structure of software reflects
the communication structure of people writing it.
It is of interest to understand how different modules present different
structure, whether or not that is associated with their contribution practice,
%? what is module structure? need a definition?
% contribution practice of different modules of Linux kernel differs from each other,
%?(evolve? business?) and how they evolve over time adapting to different business environments.
In particular, we aim to answer the following questions:
\begin{itemize}
\item What are the structures of different modules of Linux kernel?
\item Do different modules of Linux kernel have different contribution practice from each other?
%? \item Do different modules of Linux kernel evolve their practice over time and how?
%? \item Is there any relavance between module structures and contribution practice in Linux kernel?
%? participant feature: contribution type, and objective of contribution
%? evidence on commercial participation.

\end{itemize}
On the one hand,
it may help us understand the new challenges in the new FLOSS landscape like
commercial participation. %? how
On the other hand, the understanding can help us
utilize the best practices and amplify the effect.

%In our study, the three aspects we investigate are all crucial for hybrid
%project development and the results of the measures we construct are agreed
%with reality, suggesting their validity.

%By using code change history, we quantify how contribution is organized for this FLOSS project.
We retrieve the code commits from the mainline repository of linux kernel, and
use the data to analyze module structure and quantify the community contribution practice.
We observed that different modules of Linux kernel have different structures in terms of modularity.
In particular, the module of drivers is better modularized than other modules.

We focus on two particular factors that try to depict contribution practice.
The first one is the ratio of number of authors to number of committers(A2C), which tries to measure %? factor? or metrics
how many authors each committer commits code for on average, i.e. team size. This factor represents contribution organization. We found that
1), the module of drivers is unique among all the modules in terms of having the biggest team size; %? and organized more spontaneously. %? spontaneously
2), the ratio of A2C is decreasing over time for most of modules, even though both the number of authors and committers are increasing (the module of drivers is the single module that has a sharp increase for both that is different from the other modules), apparently, the increase of committers is faster than that of authors.
That may suggest a more professional organization of the working teams has been happening in the community;
3), the ratio of A2C on each module correlates with the number of new joiners, and the number of new LTCs. This may imply that a looser control of working team brings more outsiders which requires attention from the community.

%code ownership?
The second factor to depict contribution practice is code ownership, defined by number of committers per file. This factor measures how close the cooperation is. We found that the core modules of Linux kernel, like $mm$ and $kernel$, have stronger code ownership than other modules.

The rest of the paper is organized as follows. Section~\ref{s:related}
discusses the related work. Section~\ref{s:method} describes the research
methodology used in our study. Section~\ref{s:result} presents the results
of our study. Section~\ref{s:limitation} discuses limitations of our study.
We discuss practical implications of our empirical results in Section~\ref{s:discussion}
and conclude in Section~\ref{s:conclusion}.

\section{Background and Related Work}\label{s:related}

%online communities can be designed and managed to achieve the goals that their owners, managers or members desire.~\cite{Kraut2012}
%Open source software development communities have the potential to provide firms with a valuable
%platform for innovation and product development (Chesbrough 2003). In order to realize such potential
%benefits, a firm must participate in an OSS community. If such involvement is articulated through direct
%contributions by the firm's developers, the firm has to choose which software developers to dedicate to
%this effort~\cite{Daniel2011}.
The nature and performance of FLOSS development are subject of
numerous investigations, but studies on evolution of contribution practice
along the change of project landscape particularly commercial participation
 are far less common.

An early study of Apache web server and Mozilla web browser~\cite{MFH02}
quantified various aspects of OSS development practices. The
results were framed as seven hypotheses that outline key aspects of
OSS development.  In this paper we focus on a different aspect of
 contribution organization.

Community strategies and practices are often addressed in the
literature, e.g., community architecture~\cite{MFH02,YK03}, license
and intelligence property management mechanism~\cite{Hippel03}, and
code commit privilege or ownership control~\cite{MFH02,YK03,KSL03}, etc.
  Meneely  and  Williams~\cite{meneely09}  examined  the  relationship
of  the  number  of  developers  working  on  parts  of  the
Linux kernel with security vulnerabilities.  They found
that when more than nine developers contribute to a source
file, it is sixteen times more likely to include a security vulnerability.

Unlike in prior work, we observe the contribution practice
presented by Linux kernel experiencing various of technical and economic landscape
and quantify the important aspects of the development that are likely
to be affected by the product structure:
 the organization of author to committer team.

According to Conway's law~\cite{conway}, the structure of software reflects
the communication structure of people writing it.
Conway's law emphasizes the effect on the artifacts induced by social activities,
and provides insights on how to look at software development
through the perspectives of organizational science.

\section{Methodology}\label{s:method}
\subsection{Data preparation}
We retrieved all the changes from the mainline repository of Linux
kernel\footnote{git://git.kernel.org/pub/scm/linux/kernel/git/torvalds/linux.git}.
We took steps to clean and standardize the data, and obtained a data level for the further analysis.
According to the following rules, we obtained the final changes for this study:
1), we only look at changes to $.c$ files. We consider $.c$ files to be important files as they are implementation of functionality in Linux kernel. These changes account for 65.86\% of all changes.
2), we consider the first level directory retrieved from the file path as the module used in Linux kernel. Overall we obtained 22 modules. The top seven modules, in terms of number of changes, include drivers, arch(architecture), net, fs(file system), sound, kernel, mm(memory management).
%? has nothing to do with this study 3), we consider a developer's joining time as the time they author their first commit.

Table~\ref{tab:data} shows an observation used for this study.
Each observation corresponds to a change. It records who and when writes the code,
who and when commits the code, and which file this change is made on.
The author and the committer may be the same person, but most of the time (74\%) they are not.
The changed file is represented by its directory in the code repository.
For example, ``drivers/pci/iova.c'' illustrates that the changed file
is under the directory of pci driver. The last column named module in the table is
the first level directory retrieved from the file path.

\begin{table*}
\centering
\caption{Attributes of an observation}
\begin{tabular}{c|c|c|c|c|l} \hline
author & author time & committer & commit time & changed file & module\\ \hline
 Minfei Huang & Nov 6 16:32:45 2015 -0800 & Linus Torvalds & Nov 6 17:50:42 2015 -0800 & kernel/kexec.c &kernel\\ \hline
\label{tab:data}
\end{tabular}
\end{table*}
\section{Note}\label{s:note}


\section{Discussion}\label{s:discussion}

\section{Limitation}\label{s:limitation}

\section{Conclusion}\label{s:conclusion}

\bibliographystyle{abbrv}
\bibliography{../paper/bib/audris,../paper/bib/all,../paper/bib/pkuas,../paper/bib/hybrid}


\end{document}
